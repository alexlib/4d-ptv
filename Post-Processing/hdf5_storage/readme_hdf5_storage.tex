\documentclass{article}
\usepackage{siunitx}

\title{HDF5 storage toolbox}
\author{Thomas Basset}

\begin{document}

\maketitle

\section{Why to use it?}
We organize track data with two formats: a work format which is a MATLAB structure (easy to use but not optimal to save and difficult to read with other languages), and a storage format which is a HDF5 file (easy to save and read). Thus this MATLAB toolbox enables to save a MATLAB track structure in a normalized format in a HDF5 file, and vice versa, to generate the initial MATLAB track structure from the HDF5 file.

\section{How to use it?}
As an example, we use the file \textit{tracks\_sample.mat}: 1473 stitched tracks longer than 50 frames (6250 fps) of tracers (\SI{250}{\micro\meter}) from homogeneous isotropic turbulence in water (LEM experiments). They are saved in the work format: a structure with one line per track and different fields for different quantities (here \textit{x,y,z} for the positions, \textit{vx,vy,vz} for the velocities, and \textit{t} for the frame number), and a field \textit{L} for the length of each track. To save them in the storage format, the tracks are concatenated in one big array for each field, then each field is saved as a dataset in the .h5 file. To get back to the initial structure, the field \textit{L} enables to rearrange per track by slicing the big array. The script \textit{run\_h5\_storage} gives a run example.

\section{Functions}
\textit{help function name} gives some documentation, especially input and output arguments. These functions are commented and designed to be easily modified.
\begin{itemize}
\item \textit{tracks\_h52mat}: read the .h5 file generated by \textit{tracks\_mat2h5} and return the initial track structure
\item \textit{tracks\_mat2h5}: save a track structure in .h5 format
\end{itemize}

\end{document}